\documentclass{article}

\usepackage{geometry}

\title{Scheduling basics with MOSS Scheduling Simulator -- an EOPSY laboratory}
\author{Maciej Marcinkiewicz}
\date{17th May 2021}

\newgeometry{lmargin=3.2cm, rmargin=3.2cm, bmargin=2.5cm}

\begin{document}

\maketitle

\section{Introduction}
\subsection{Brief description of the task}
The main purpose of this laboratory was to get students familiarized with the basics of
process scheduling in operating systems. For that purpose MOSS Scheduling Simulator had to
be used. The task was to run 2, 5 and 10 processes in 10000 ms long simulation and
to analyse simulation's output data.

In the simulator one may specify number of processes, the mean and standard deviation
for compute time, I/O blocking time for each process and the duration time of whole simulation.

\subsection{Environment description}
In this laboratory simulated situation is very simple -- \emph{first come, first serve}
(FCFS) algorithm is being used. The process that is on the top of
ready queue is run for desired amount of time without any pausing. It is a \emph{non-preemptive scheduler}, so the simulated
operating system cannot interrupt processes in order to allow another process of higher priority for
execution. Instead the tasks are being done for such long time it is needed by the current process.
Such operating systmes are called \emph{batch operating systems}.

\subsection{Environment configuration}
Each simulation had the same parameters. Mean deviation time was set to 2000 ms whereas
standard deviation was set to 0 ms. It means that each simulated process was running
for 2000 ms before termination and no randomness of that time was supplied.
Each process had 500 ms of I/O blocking, so processes after each 500 ms of running
were blocking input or output. The last parameter, runtime, was set to 10000 ms.
It was the maximal possible time of execution for the whole simulation.

\section{Simulation of 2 processes}

\end{document}