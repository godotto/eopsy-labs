\documentclass{article}

\usepackage{geometry}
\usepackage{tabularx,array}

\title{Scheduling basics with MOSS Scheduling Simulator -- an EOPSY laboratory}
\author{Maciej Marcinkiewicz}
\date{17th May 2021}

\newgeometry{lmargin=3.2cm, rmargin=3.2cm, bmargin=2.5cm}

\begin{document}

\maketitle

\section{Introduction}
\subsection{Brief description of the task}
The main purpose of this laboratory was to get students familiarized with the basics of
process scheduling in operating systems. For that purpose MOSS Scheduling Simulator had to
be used. The task was to run 2, 5 and 10 processes in 10000 ms long simulation and
to analyse simulation's output data.

In the simulator one may specify number of processes, the mean and standard deviation
for compute time, I/O blocking time for each process and the duration time of whole simulation.

\subsection{Environment description}
In this laboratory simulated situation is very simple -- \emph{first come, first serve}
(FCFS) algorithm is being used. The process that is on the top of
ready queue is run for desired amount of time without any pausing. It is a \emph{non-preemptive scheduler}, so the simulated
operating system cannot interrupt processes in order to allow another process of higher priority for
execution. Instead the tasks are being done for such long time it is needed by the current process.
Such operating systmes are called \emph{batch operating systems}.

\subsection{Environment configuration}
Each simulation had the same parameters. Mean deviation time was set to 2000 ms whereas
standard deviation was set to 0 ms. It means that each simulated process was running
for 2000 ms before termination and no randomness of that time was supplied.
Each process had 500 ms of I/O blocking, so processes after each 500 ms of running
were blocking input or output. The last parameter, runtime, was set to 10000 ms.
It was the maximal possible time of execution for the whole simulation.

\newpage

\section{Simulation of 2 processes}
\subsection{Simulator's output}
\subsubsection*{Summary -- results}
\ttfamily

Scheduling Type: Batch (Nonpreemptive)\\
Scheduling Name: First-Come First-Served\\
Simulation Run Time: 4000\\
Mean: 2000\\
Standard Deviation: 0\\
\begin{tabular}{@{}l l l l l}
    Process \# &	CPU Time &	IO Blocking &	CPU Completed &	CPU Blocked\\
    0		    &2000 (ms)	&500 (ms)	&2000 (ms)	    &3 times\\
    1		    &2000 (ms)	&500 (ms)	&2000 (ms)	    &3 times
\end{tabular}

\subsubsection*{Summary -- processes}
Process: 0 registered... (2000 500 0)\\
Process: 0 I/O blocked... (2000 500 500)\\
Process: 1 registered... (2000 500 0)\\
Process: 1 I/O blocked... (2000 500 500)\\
Process: 0 registered... (2000 500 500)\\
Process: 0 I/O blocked... (2000 500 1000)\\
Process: 1 registered... (2000 500 500)\\
Process: 1 I/O blocked... (2000 500 1000)\\
Process: 0 registered... (2000 500 1000)\\
Process: 0 I/O blocked... (2000 500 1500)\\
Process: 1 registered... (2000 500 1000)\\
Process: 1 I/O blocked... (2000 500 1500)\\
Process: 0 registered... (2000 500 1500)\\
Process: 0 completed... (2000 500 2000)\\
Process: 1 registered... (2000 500 1500)\\
Process: 1 completed... (2000 500 2000)

\rmfamily

\subsection{Comments}
The simulation lasted for 4000 ms -- 2000 ms for each of two processes. Both processes
has been executed properly and were given required time. The way of processing them
was very simple and repetitive. Starting from the beginning, process 0 has been acknowledged
by the scheduler and was running 500 ms until it blocked its I/O.

Blocked processes are unable to run until some external event happens, so it means that
another process from queue can be run instead. After blocking process 0, the same pattern
is happening to process 1 -- it runs for 500 ms and then it blocks I/O.

While process 1 is blocked, process is ready again, it is once again reqistered by scheduler
and it starts its execution. Same as previously after 500 ms it blocks its I/O and then
process 1 is registered by the scheduler. Whole pattern is repeated until processes'
total time reaches the required 2000 ms. After such amount of time process does not
blocks its I/O and it terminates (which is represented as \emph{completed} message from
simulator).

Both processes has been executed in four, 500 ms long intervals. They blocked for input
3 times each. After last 500 ms they did not block but just terminated.

\section{Simulation of 5 processes}
\subsection{Simulator's output}
\subsubsection*{Summary -- results}
\ttfamily
Scheduling Type: Batch (Nonpreemptive)\\
Scheduling Name: First-Come First-Served\\
Simulation Run Time: 10000\\
Mean: 2000\\
Standard Deviation: 0\\
\begin{tabular}{@{}l l l l l}
    Process \#&	CPU Time&	IO Blocking&	CPU Completed&	CPU Blocked\\
    0		&2000 (ms)	&500 (ms)	&2000 (ms)	&3 times\\
    1		&2000 (ms)	&500 (ms)	&2000 (ms)	&3 times\\
    2		&2000 (ms)	&500 (ms)	&2000 (ms)	&3 times\\
    3		&2000 (ms)	&500 (ms)	&2000 (ms)	&3 times\\
    4		&2000 (ms)	&500 (ms)	&2000 (ms)	&3 times
\end{tabular}

\subsubsection*{Summary -- processes}
Process: 0 registered... (2000 500 0 0)\\
Process: 0 I/O blocked... (2000 500 500 500)\\
Process: 1 registered... (2000 500 0 0)\\
Process: 1 I/O blocked... (2000 500 500 500)\\
Process: 0 registered... (2000 500 500 500)\\
Process: 0 I/O blocked... (2000 500 1000 1000)\\
Process: 1 registered... (2000 500 500 500)\\
Process: 1 I/O blocked... (2000 500 1000 1000)\\
Process: 0 registered... (2000 500 1000 1000)\\
Process: 0 I/O blocked... (2000 500 1500 1500)\\
Process: 1 registered... (2000 500 1000 1000)\\
Process: 1 I/O blocked... (2000 500 1500 1500)\\
Process: 0 registered... (2000 500 1500 1500)\\
Process: 0 completed... (2000 500 2000 2000)\\
Process: 1 registered... (2000 500 1500 1500)\\
Process: 1 completed... (2000 500 2000 2000)\\
Process: 2 registered... (2000 500 0 0)\\
Process: 2 I/O blocked... (2000 500 500 500)\\
Process: 3 registered... (2000 500 0 0)\\
Process: 3 I/O blocked... (2000 500 500 500)\\
Process: 2 registered... (2000 500 500 500)\\
Process: 2 I/O blocked... (2000 500 1000 1000)\\
Process: 3 registered... (2000 500 500 500)\\
Process: 3 I/O blocked... (2000 500 1000 1000)\\
Process: 2 registered... (2000 500 1000 1000)\\
Process: 2 I/O blocked... (2000 500 1500 1500)\\
Process: 3 registered... (2000 500 1000 1000)\\
Process: 3 I/O blocked... (2000 500 1500 1500)\\
Process: 2 registered... (2000 500 1500 1500)\\
Process: 2 completed... (2000 500 2000 2000)\\
Process: 3 registered... (2000 500 1500 1500)\\
Process: 3 completed... (2000 500 2000 2000)\\
Process: 4 registered... (2000 500 0 0)\\
Process: 4 I/O blocked... (2000 500 500 500)\\
Process: 4 registered... (2000 500 500 500)\\
Process: 4 I/O blocked... (2000 500 1000 1000)\\
Process: 4 registered... (2000 500 1000 1000)\\
Process: 4 I/O blocked... (2000 500 1500 1500)\\
Process: 4 registered... (2000 500 1500 1500)

\rmfamily

\subsection{Comments}
Once again each process has been properly executed as total required time fits perfectly
given max time of 10000 ms for simulation (it is just 5 times 2000 ms). Scheduler choose
processes in the same manner as last time. However, after the first blocking of process 1's
I/O, scheduler registered process 0, not process 2 as it might seem more resonable at the
first glance. 

The reason for that is the way scheduling is done in simulation -- FCFS.
When process blocks, process 0 is unblocked and ready to be run. As it is the first in queue,
the scheduler chooses it to be registered instead of process 2. Because of that,
processes 2 and 3 had been executed when 0 and 1 were terminated. The same applies for the last,
5th process -- it has been registered after execution of processes 2 and 3 was completed.

One more interesting remark is that even in \emph{summary -- results} it was told that
process 4 was running for required 2000 ms of time and still there was no trace of process'
termination. It happend probably most likely due to simulation termination exactly after 10000 ms
as it was specifed in configuration. It did not have any time left for proper termination.

\end{document}